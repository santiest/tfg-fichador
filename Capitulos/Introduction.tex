\chapter{Introduction and objectives}
\label{cap:introduction}
% \addcontentsline{toc}{chapter}{Introduction}


Recent legislative changes introduced by both the European Union and the Spanish government have mandated 
alterations to employee clocking procedures. Notably, Spain now requires employees to clock in upon arrival 
at work, necessitating the implementation of reliable time-tracking systems. Additionally, the European Union 
has banned the use of clocking devices employing biometric identification methods such as fingerprint 
scanning or facial recognition.

Acciona Facility Services manages many thousands of employees throughout Spain, and it was of utmost importance
to expeditely replace the non-compliant devices for different ones, as many companies were already being
imposed hefty fines, from tens to hundreds of thousands of Euros, for not meeting the new regulatory 
requirements.

The company already had providers for many types of clocking devices, and many of them were compliant, but 
were priced in the hundreds of Euros each. Now, faced with having to replace them, the prospect of replacing 
\textit{thousands} of units at considerable expense loomed large.  Moreover, outsourcing these devices often 
meant committing to complex time-tracking ecosystems, adding further complications and increasing the
complexity of the data the company manages.

These issues were brought to light to the innovation team at Acciona Facility Services, and were then tasked
with bringing a solution to market.



