\chapter{Conclusions and Future Work}
\label{cap:conclusions}

The project has been a great success and is currently being rolled out to production. This 
achievement marks a significant milestone in the pursuit of creating a more cost-effective and 
durable solution for employee clocking systems.

In conclusion, the development process—from the initial concept through to the final 
deployment—has demonstrated the effectiveness of meticulous planning, innovative design, and 
iterative testing. Key accomplishments of the project include:

\begin{itemize}
    \item \textbf{Cost Efficiency}: The newly developed system significantly reduces costs 
    compared to the previous solution using mobile phones. By utilizing a Raspberry Pi Pico and 
    custom-designed enclosures, the project has lowered both initial and long-term expenses.
    \item \textbf{Enhanced Longevity}: The robust design of the new device promises a much longer 
    lifespan than the mobile phones previously used. The components are expected to last at least 
    three years, and in case of any failure, individual parts can be replaced rather than 
    discarding the entire device.
    \item \textbf{Customizability and Flexibility}: The modular nature of the system allows for 
    easy upgrades and customization. This adaptability ensures the system can evolve with changing 
    technological requirements and company needs.
    \item \textbf{Improved Reliability}: The new enclosure and internal design provide superior 
    protection against environmental factors, such as heat and impact, which plagued the mobile 
    phone-based solution.
    \item \textbf{Advanced Functionality}: The integration with Google Cloud Platform and the use 
    of Spring for API development have ensured a seamless and secure data transmission process, 
    enhancing the overall functionality and reliability of the system.
\end{itemize}

The project's success underscores the importance of selecting appropriate materials, technologies, 
and design methodologies. The choice of PETG for the enclosure, the adoption of 3D printing for 
rapid prototyping, and the iterative design process have all contributed to the creation of a 
highly effective solution.

As the rollout continues, ongoing monitoring and feedback will be essential to ensure the system 
meets all operational requirements and continues to perform as expected. Future improvements may 
include further optimization of the design, enhancements in software functionality, and potential 
expansion of the system's capabilities.

This thesis not only documents the journey from concept to deployment but also serves as a guide 
for similar projects aiming to leverage technology for practical, cost-effective solutions. The 
insights gained and the methodologies developed during this project can be applied to a wide range 
of applications beyond employee clocking systems, demonstrating the broader impact and potential 
of this work.


\section{Future Work}

The development of this project is still ongoing, with new improvements being added on a daily 
basis. There is a comprehensive roadmap for future development, including implementing remote 
updates for the device's code, which will significantly enhance its maintainability and 
flexibility.

Very shortly, the devices will be updated to periodically send ``keepalive'' messages to the server. 
These messages will not only confirm that the device is operational but will also report any 
ongoing hardware issues. This proactive approach will enable quick identification and resolution 
of potential problems, ensuring the devices operate smoothly and reliably.

However, several challenges remain. The limited memory capacity of the Raspberry Pi Pico and the 
intermittent connectivity issues with Google Cloud have been significant obstacles. Addressing 
these problems is not straightforward, but overcoming them could unlock immense future 
opportunities.

The optimal solution to these issues involves designing a custom PCB that integrates both the 
microcontroller and the cellular connectivity module. This ambitious endeavor would require months 
of meticulous planning, prototyping, and testing, but it would represent a substantial upgrade to 
the current system.

Additionally, it is likely that the microcontroller will need to be upgraded to a model with 
greater memory capacity, such as the ESP32-S3. The ESP32-S3 not only offers more memory but also 
provides similar functionality, making it a suitable replacement for the Raspberry Pi Pico.

While the process of designing this integrated PCB has not yet begun, and discussions about its 
feasibility are ongoing, it is clear that such a development would greatly enhance the device's 
capabilities. This future work promises to further improve the reliability, efficiency, and 
overall performance of the system, building on the significant advantages already realized in the 
current project.
