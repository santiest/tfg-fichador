\chapter{Introducción y Objetivos}
\label{cap:introduccion}

Cambios recientes en la legislación introducidos tanto por la Agencia Española de Protección de 
Datos como por el gobierno nacional han exigido alteraciones en los procedimientos de fichaje 
de los empleados. En particular, España ahora requiere que los empleados registren su entrada al 
llegar al trabajo, lo que obliga a implementar sistemas confiables de seguimiento del tiempo 
\cite{boe_obligacionfichajes}. Además, la Agencia Española de Protección de Datos ha prohibido el 
uso de dispositivos de fichaje que empleen métodos de identificación biométrica, como el escaneo 
de huellas dactilares o el reconocimiento facial \cite{aepd_prohibicionbiometricos}.

Una empresa en particular gestiona muchos miles de empleados en toda España, y era de suma 
importancia reemplazar rápidamente los dispositivos no conformes por otros diferentes, ya que 
muchas empresas ya estaban siendo multadas con sumas considerables, desde decenas hasta cientos de 
miles de euros, por no cumplir con los nuevos requisitos regulatorios.

La empresa ya tenía proveedores para muchos tipos de dispositivos de fichaje, y muchos de ellos 
eran conformes, pero tenían un precio de varios cientos de euros cada uno. Ahora, enfrentándose a 
la necesidad de reemplazar \textit{miles} de unidades a un costo considerable, la perspectiva de 
hacerlo se presentaba como un gran desafío. Además, subcontratar estos dispositivos a menudo 
implicaba comprometerse con ecosistemas complejos de administración de empleados, lo que añadía 
más complicaciones e incrementaba la complejidad de los datos que la empresa maneja.

Estos problemas fueron planteados al equipo de innovación de la empresa, que fue encargado de 
llevar una solución al mercado.

\section{Objetivos}

Dados los desafíos y consideraciones anteriores, se consideró necesario el desarrollo de un 
dispositivo rentable y el establecimiento de una infraestructura en la nube.

Los objetivos de este proyecto son los siguientes:

\begin{itemize}
	\item \textbf{Diseño y Prototipado}: Diseñar un dispositivo utilizando microcontroladores y 
	módulos electrónicos adicionales, como pantallas LCD y lectores NFC, para cumplir con los 
	requisitos especificados.
	\item \textbf{Diseño de PCB}: Desarrollar un diseño de placa de circuito impreso (PCB) para 
	facilitar la interconexión de los diversos módulos electrónicos utilizados en el dispositivo, 
	asegurando un rendimiento eficiente y confiable.
	\item \textbf{Desarrollo de API}: Crear una \textit{Application Programming Interface} (API) 
	capaz de desplegarse como un contenedor de Docker o una Google Cloud Function, permitiendo una 
	comunicación fluida entre el dispositivo y los servicios basados en la nube.
	\item \textbf{Investigación de Microcontroladores}: Investigar las limitaciones y capacidades 
	de los microcontroladores, particularmente los basados en RP2040, para informar las decisiones 
	de diseño y optimizar el rendimiento.
	\item \textbf{Impresión y Modelado 3D}: Explorar los fundamentos de la impresión 3D y el 
	modelado 3D necesarios para diseñar una caja adecuada para el dispositivo. Además, proporcionar 
	una visión general y comparación de varios materiales adecuados para la fabricación de la 
	caja, considerando factores como durabilidad, costo y atractivo estético.
	\item \textbf{Análisis de Costos}: Realizar una breve comparación de costos entre la nueva 
	solución y el sistema anterior, demostrando los beneficios financieros del nuevo diseño.
\end{itemize}