\chapter{API Development}
\label{cap:apiDevelopment}

After developing a basic working device, the next major milestone was enabling it to send data 
over the internet. Achieving this milestone was crucial for demonstrating the project's 
progress to executives, proving that the project is on track for success. Such progress not only 
instills confidence in the project's viability but also helps secure additional financing for 
further development. This financing is essential for acquiring necessary resources, such as a 3D 
printer, which will be discussed in the next chapter.

Developing an \textit{Application Programming Interface} (API) was a fundamental step in this 
process. The API acts as an intermediary, facilitating communication between the device and a 
database where clockings are stored. It ensures that data from the device is accurately and 
efficiently transmitted to the database. This capability is vital for the real-time tracking and 
management of employee clockings, which forms the core functionality of the device.

This approach adds an extra layer of security. The alternative—directly inserting clockings into 
the database—would pose a significant security risk. If someone were to open the device and access 
its code, they could potentially see the database credentials. Given that the database contains 
personal data, maintaining its security is paramount. By using an API, the devices are restricted 
to only sending data, without the ability to read any data. This ensures that even if the device 
is compromised, the database credentials remain protected and the integrity of the personal data 
is maintained.