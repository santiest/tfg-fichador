\chapter{Breadboard Prototype}
\label{cap:breadboardPrototype}

The first phase of developing the new clocking device involved thorough research into available components and 
microcontrollers in the market. Factors such as cost, functionality, and potential drawbacks were carefully 
evaluated to inform the selection process. Subsequently, selected components were tested by constructing a 
basic prototype on a breadboard to assess functionality and performance. Demonstrating the viability of the 
project at this stage was crucial for ensuring its continued development and success.

\section{Hardware selection}

Firstly, considering that the development was urgent, it would only be possible to use an already made 
controller. There were two routes to take:
\begin{itemize}
	\item Using a \textbf{single-board computer}.
	\item Using a \textbf{microcontroller}.
\end{itemize}

The team's familiarity with \textit{Raspberry Pi} products and their reputation for reliability led to the 
decision to utilize their offerings for the project. Given the lightweight processing requirements, the 
\textit{Raspberry Pi Zero 2 W}, a cost-effective single-board computer with built-in WiFi capabilities, emerged 
as a suitable option. Alternatively, the \textit{Raspberry Pi Pico W}, a microcontroller, presented another 
viable choice.

However, it's important to note that while the \textit{Pi Zero} offers more features and functionality, it 
comes at a higher cost compared to the \textit{Pi Pico}. In fact, it costs almost three times as much. 
Therefore, careful consideration is warranted to determine whether the additional expense justifies the benefits.

\begin{figure}[h]
    \centering
    \begin{minipage}[b]{0.45\textwidth}
        \centering
        \includegraphics[width=.6\textwidth]{Imagenes/Vectorial/piZero2W.pdf}
        \caption{Raspberry Pi Zero 2 W}
        \label{fig:piZero2W}
    \end{minipage}
    \hfill
    \begin{minipage}[b]{0.45\textwidth}
        \centering
        \includegraphics[width=.6\textwidth]{Imagenes/Vectorial/piPicoWH.pdf}
        \caption{Raspberry Pi Pico WH}
        \label{fig:piPicoWH}
    \end{minipage}
\end{figure}


%
% Single-board Computers
%
\subsection{Single-board Computers}

A single-board computer (\textit{SBC}) is a complete computer built on a single circuit board. It integrates all 
the necessary components required for a functional computer system, including a central processing unit (CPU), 
memory (RAM), storage (usually in the form of a MicroSD card), input/output ports, and sometimes additional 
features such as networking capabilities (e.g., Ethernet or WiFi), audio/video output, GPIO (General Purpose 
Input/Output) pins for connecting external devices, and even USB ports.

SBCs are designed to be compact and efficient, which is the case of the \textit{Pi Zero}, measuring about 65mm by 
30mm while drawing about one Watt of power. Additionally, SBCs can run an operating system, such as Ubuntu.

This ability to run an entire operating system significantly enhances their versatility compared to microcontrollers 
out-of-the-box. Many peripheral devices can simply be plugged into a USB port and function seamlessly without 
requiring additional configuration. For instance, a 3G SIM card adapter, which will be necessary for future stages 
of the project, can be effortlessly integrated into the system, letting the operating system take charge of the 
communication with the mobile network, abstracting all these problems from the programmer.


%
% Microcontrollers
%
\subsection{Microcontrollers}

A microcontroller is a compact integrated circuit (IC) that contains a central processing unit (CPU), memory (both 
volatile RAM and non-volatile flash memory), input/output peripherals (such as digital and analog I/O pins), and 
various other hardware components necessary for interfacing with external devices. Unlike single-board computers, 
microcontrollers are typically designed for specific tasks and embedded applications, often with real-time requirements.

One key characteristic of microcontrollers is their ability to execute dedicated firmware or software code stored in 
their internal memory. This code typically controls the behavior of the microcontroller, processes inputs from sensors 
or other external devices, and generates outputs to control actuators or display information.

The \textit{Raspberry Pi Pico W} is a development board that utilizes the \textit{RP2040} microcontroller chip. This board 
offers a range of features beyond its microcontroller, including onboard flash memory for program storage, versatile 
GPIO pins for interfacing with external devices, built-in USB connectivity for programming and power supply, and 
WiFi connectivity.

In comparison to single-board computers, the \textit{Pi Pico} does not have an operating system. Instead, firmware can be 
loaded onto it, and it is this firmware that provides functionality to the board. This firmware allows the programmer to 
use programming languages such as C or MicroPython, which then control the microcontroller's behavior and interactions with 
external devices.

The absence of an operating system reduces the overhead associated with system management and resource allocation, resulting 
in faster boot times and improved reliability for time-critical tasks.

Taking into account our previously established requirements, which prioritize minimal points of failure, low computational 
demands, and cost-effectiveness, the logical preference leans towards the utilization of a microcontroller. For instance, 
single-board computers often rely on SD cards for storage, which can be prone to failure after prolonged use due to factors 
such as wear and tear or data corruption. In contrast, microcontrollers typically have simpler storage mechanisms, such as 
onboard flash memory, which are less susceptible to such issues. Thus, lower operating costs.

As a conclusion, a \textbf{microcontroller will be used}, and in particular, the \textit{\textbf{Raspberry Pi Pico W}}.

%
% Electronic Modules
%
\subsection{Electronic Modules}

Now that the microcontroller has been selected, additional components are necessary to meet the project's requirements. 
Specifically, the device must integrate an NFC reader and an LCD screen. Additionally, other peripherals, such as a buzzer 
or LED, may also be evaluated.

We need to take into account the available buses and choose modules accordingly. Relying solely on the datasheet alone is 
not enough to determine the buses that are usable concurrently, since if two different buses use the same pin, then they 
cannot be used simultaneously. Referring to the pinout diagram of the \textit{Raspberry Pi Pico W} provided below, we can 
identify the available buses:

\begin{figure}[h]
	\centering
	\includegraphics[width = 1\textwidth]{Imagenes/Vectorial/picow-pinout.pdf}
	\caption{Raspberry Pi Pico W's Pinout}
	\label{fig:piPicoPinout}
\end{figure}

For example, each UART bus conflicts with I2C buses. Therefore, when selecting components, it's crucial to ensure 
compatibility with this layout.


\subsubsection*{NFC Module}

Various NFC boards are available on the market. Ideally, the desired NFC board would offer extended range, compact 
form-factor, low power consumption, and support for multiple communication protocols such as UART, I2C, SPI, among others.

The \textit{PN532} chipset produced by \textit{NXP} offers all three types of buses mentioned before, that is: High Speed 
UART, I2C and SPI. Numerous boards utilizing this chipset are available on the market, with \textit{ElecHouse}'s offering 
standing out for its excellent form-factor (about 4cm$\times$4cm) and impressive range. This specific board has been 
extensively replicated and is accessible at a low cost, and various providers offer open-source code for controlling it.

\begin{figure}[h]
	\centering
	\includegraphics[width = 0.3\textwidth]{Imagenes/Vectorial/PN532.pdf}
	\caption{PN532 board}
	\label{fig:pn532}
\end{figure}

\subsubsection*{LCD Module}

Initially, the idea of utilizing a black and white OLED display seemed appealing. However, as the size increased, so did 
the price, and even then, the displays were still too small. Consequently, the most practical decision was to adhere to 
LCD technology.

Upon researching LCD modules, the \textit{LCD1602} module emerged as a suitable choice. It offers sufficient size, 
accommodating up to 16 characters per row across 2 rows, and provides satisfactory contrast. Additionally, it is 
available with I2C adapters for simplified control, requiring fewer pins on the microcontroller.

\begin{figure}[h]
	\centering
	\includegraphics[width = 0.7\textwidth]{Imagenes/Vectorial/LCD1602.pdf}
	\caption{LCD1602 board}
	\label{fig:lcd1602}
\end{figure}


\subsubsection*{Buzzer}

Beyond just relying on visual cues displayed on the screen to convey the device's status, it's essential to enhance 
the user experience by incorporating auditory feedback. When employees clock in to work, providing a distinct sound 
signal from a buzzer ensures they receive immediate confirmation of their action.

There are two types of buzzer:
\begin{itemize}
	\item \textbf{Active buzzers}: they incorporate an internal oscillator circuit that generates the sound signal 
	when voltage is applied. They are self-contained and do not require external circuitry to produce sound. They 
	are commonly used in applications where simplicity and ease of use are prioritized, as they can be directly 
	connected to a power source to emit sound.
	\item \textbf{Passive buzzers}: they require an external oscillating circuit to produce sound. An alternating 
	current is applied to create vibrations that produce sound waves. They offer more flexibility in sound frequency 
	and intensity control but require additional circuitry for operation.
\end{itemize}

Due to the project's time constraints and the preference for simplicity, \textbf{active buzzers will be employed}. 
Additionally, since a single sound frequency suffices for the intended use, active buzzers are well-suited for the task.

% \begin{figure}[h]
% 	\centering
% 	\includegraphics[width = 0.5\textwidth]{Imagenes/Vectorial/Todo.pdf}
% 	\caption{Ejemplo de imagen}
% 	\label{fig:sampleImage}
% \end{figure}

% Si te sirve de utilidad,  puedes incluir tablas para mostrar resultados, tal como se ve en la tabla \ref{tab:sampleTable}.


% \begin{table}
% 	\centering
% 	\begin{tabular}{c|c|c}
% 		\textbf{Col 1} & \textbf{Col 2} & \textbf{Col 3} \\
% 		\hline\hline
% 		3 & 3.01 & 3.50\\
% 		6 & 2.12 & 4.40\\
% 		1 & 3.79 & 5.00\\
% 		2 & 4.88 & 5.30\\
% 		4 & 3.50 & 2.90\\
% 		5 & 7.40 & 4.70\\
% 		\hline
% 	\end{tabular}
% 	\caption{Tabla de ejemplo}
% 	\label{tab:sampleTable}
% \end{table}
