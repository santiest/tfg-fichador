\chapter{Breadboard Prototype}
\label{cap:breadboardPrototype}

The first phase of developing the new clocking device involved thorough research into available components and 
microcontrollers in the market. Factors such as cost, functionality, and potential drawbacks were carefully 
evaluated to inform the selection process. Subsequently, selected components were tested by constructing a 
basic prototype on a breadboard to assess functionality and performance. Demonstrating the viability of the 
project at this stage was crucial for ensuring its continued development and success.

\section{Hardware selection}

Firstly, considering that the development was urgent, it would only be possible to use an already made 
controller. There were two routes to take:
\begin{itemize}
	\item Using a \textbf{single-board computer}.
	\item Using a \textbf{microcontroller}.
\end{itemize}

The team's familiarity with \textit{Raspberry Pi} products and their reputation for reliability led to the 
decision to utilize their offerings for the project. Given the lightweight processing requirements, the 
\textit{Raspberry Pi Zero W}, a cost-effective single-board computer with built-in WiFi capabilities, emerged 
as a suitable option. Alternatively, the \textit{Raspberry Pi Pico W}, a microcontroller, presented another 
viable choice.

However, it's important to note that while the \textit{Pi Zero} offers more features and functionality, it 
comes at a higher cost compared to the \textit{Pi Pico}. In fact, it costs almost three times as much. 
Therefore, careful consideration is warranted to determine whether the additional expense justifies the benefits.

%
% Single-board Computers
%
\subsection{Single-board Computers}

A single-board computer (\textit{SBC}) is a complete computer built on a single circuit board. It integrates all 
the necessary components required for a functional computer system, including a central processing unit (CPU), 
memory (RAM), storage (usually in the form of a MicroSD card), input/output ports, and sometimes additional 
features such as networking capabilities (e.g., Ethernet or WiFi), audio/video output, GPIO (General Purpose 
Input/Output) pins for connecting external devices, and even USB ports.

SBCs are designed to be compact and efficient, which is the case of the \textit{Pi Zero}, measuring about 65mm by 
30mm while drawing about one Watt of power. Additionally, SBCs can run an operating system, such as Ubuntu.

This ability to run an entire operating system significantly enhances their versatility compared to microcontrollers 
out-of-the-box. Many peripheral devices can simply be plugged into a USB port and function seamlessly without 
requiring additional configuration. For instance, a 3G SIM card adapter, which will be necessary for future stages 
of the project, can be effortlessly integrated into the system, letting the operating system take charge of the 
communication with the mobile network, abstracting all these problems from the programmer.


%
% Microcontrollers
%
\subsection{Microcontrollers}

A microcontroller is a compact integrated circuit (IC) that contains a central processing unit (CPU), memory (both 
volatile RAM and non-volatile flash memory), input/output peripherals (such as digital and analog I/O pins), and 
various other hardware components necessary for interfacing with external devices. Unlike single-board computers, 
microcontrollers are typically designed for specific tasks and embedded applications, often with real-time requirements.

One key characteristic of microcontrollers is their ability to execute dedicated firmware or software code stored in 
their internal memory. This code typically controls the behavior of the microcontroller, processes inputs from sensors 
or other external devices, and generates outputs to control actuators or display information.

The \textit{Raspberry Pi Pico W} is a development board that utilizes the \textit{RP2040} microcontroller chip. This board 
offers a range of features beyond its microcontroller, including onboard flash memory for program storage, versatile 
GPIO pins for interfacing with external devices, built-in USB connectivity for programming and power supply, and 
WiFi connectivity.

In comparison to single-board computers, the \textit{Pi Pico} does not have an operating system. Instead, firmware can be 
loaded onto it, and it is this firmware that provides functionality to the board. This firmware allows the programmer to 
use programming languages such as C or MicroPython, which then control the microcontroller's behavior and interactions with 
external devices.

The absence of an operating system reduces the overhead associated with system management and resource allocation, resulting 
in faster boot times and improved reliability for time-critical tasks.


% \begin{figure}[h]
% 	\centering
% 	\includegraphics[width = 0.5\textwidth]{Imagenes/Vectorial/Todo.pdf}
% 	\caption{Ejemplo de imagen}
% 	\label{fig:sampleImage}
% \end{figure}

% Si te sirve de utilidad,  puedes incluir tablas para mostrar resultados, tal como se ve en la tabla \ref{tab:sampleTable}.


% \begin{table}
% 	\centering
% 	\begin{tabular}{c|c|c}
% 		\textbf{Col 1} & \textbf{Col 2} & \textbf{Col 3} \\
% 		\hline\hline
% 		3 & 3.01 & 3.50\\
% 		6 & 2.12 & 4.40\\
% 		1 & 3.79 & 5.00\\
% 		2 & 4.88 & 5.30\\
% 		4 & 3.50 & 2.90\\
% 		5 & 7.40 & 4.70\\
% 		\hline
% 	\end{tabular}
% 	\caption{Tabla de ejemplo}
% 	\label{tab:sampleTable}
% \end{table}
