\chapter{Code Implementation and Development Challenges}
\label{cap:codeAndChallenges}

The software development phase represents a significant portion of the project timeline, demanding 
meticulous attention to detail and thorough problem-solving skills. In the forthcoming sections, 
an overview of the software requirements will be provided, shedding light on how these 
requirements were effectively addressed. Additionally, detailed insights into the encountered 
challenges and the corresponding solutions devised will be explored.

One crucial decision made early on was the selection of CircuitPython as the programming language 
for the device. This decision stemmed from several key factors, including its open-source nature, 
extensive library support, and the availability of pre-existing libraries with MIT licenses, 
particularly those essential for handling NFC readers. Furthermore, CircuitPython's user-friendly 
syntax and simplified microcontroller programming paradigm were deemed advantageous compared to 
alternatives like MicroPython, contributing to smoother development workflows.

However, it's important to note that despite the benefits of CircuitPython, the project encountered 
limitations imposed by the microcontroller's memory capacity, capped at 256kB. This constraint was 
exacerbated by the use of Python, a high-level language known for its memory-intensive nature. 
Throughout the development process, various memory optimization strategies were implemented to 
mitigate these limitations. These optimizations will be thoroughly explained, providing valuable 
insights into managing resource constraints in microcontroller-based projects.

In addition to the aforementioned aspects, it's crucial to highlight the iterative nature of 
software development, where frequent testing, debugging, and refinement cycles were integral to 
achieving desired functionality and performance. This iterative approach enabled the project team 
to address emerging issues promptly and iteratively enhance the software's robustness and 
reliability.

Furthermore, a detailed examination of the software architecture and design decisions made will 
offer valuable insights into the project's development methodology and the rationale behind 
specific implementation choices.

\section{CircuitPython: Advantages and Disadvantages}

CircuitPython is an open-source programming language designed specifically for 
microcontroller-based development. Developed primarily by Adafruit Industries, CircuitPython is 
built on top of the Python programming language, offering a simplified yet powerful platform for 
programming microcontrollers. Unlike traditional embedded programming languages, CircuitPython 
abstracts many low-level complexities, making it more accessible to beginners and hobbyists while 
still providing advanced features for experienced developers.

\begin{figure}[h]
	\centering
	\includegraphics[width = .5\textwidth]{Imagenes/Vectorial/circuitpython_logo.pdf}
	\caption{CircuitPython's logo}
	\label{fig:circuitpython_logo}
\end{figure}

One of the key advantages of CircuitPython is its user-friendly syntax and high-level abstractions, 
which resemble those of Python, a language renowned for its readability and simplicity. This makes 
CircuitPython particularly attractive to beginners and educators, enabling them to quickly grasp 
programming concepts and develop projects without being bogged down by intricate syntax or complex 
setup procedures.

Furthermore, CircuitPython simplifies the process of interacting with hardware peripherals and 
sensors commonly used in embedded systems. It provides a consistent and intuitive API (Application 
Programming Interface) for accessing GPIO pins, I2C and SPI interfaces, analog inputs, and other 
hardware features, streamlining the development process and reducing the learning curve associated 
with embedded programming.

Another notable advantage of CircuitPython is its extensive library support, particularly for 
popular microcontroller boards and peripherals. Adafruit maintains a vast repository of 
CircuitPython libraries, covering a wide range of sensors, displays, actuators, communication 
modules, and other components commonly used in electronics projects\footnote{Adafruit's Library 
Bundle: \url{https://github.com/adafruit/Adafruit_CircuitPython_Bundle}}. These libraries abstract 
the complexities of interfacing with specific hardware, allowing developers to focus on application 
logic rather than low-level hardware details.

Despite its numerous advantages, CircuitPython does have some limitations and drawbacks. One 
notable limitation is its higher memory footprint compared to lower-level languages like C or 
assembly, which can be a concern for projects with strict memory constraints. Additionally, while 
CircuitPython abstracts many hardware complexities, it may sacrifice some performance compared to 
bare-metal or lower-level programming approaches.

Overall, CircuitPython offers a compelling combination of simplicity, accessibility, and 
versatility for microcontroller-based development, making it an excellent choice for a wide range 
of projects, from educational endeavors to commercial products. Its rich ecosystem of libraries, 
ease of use, and strong community support contribute to its popularity among hobbyists, educators, 
and professional developers alike\cite{circuitpython_docs}.

For this project, CircuitPython version 9.0.0 will be used.

\section{Requirements}

\section{Code Implementation}

\subsection{Asynchronous Programming}

\subsection{Main Loop and Operating Modes}

\subsection{Interchangeable WiFi and Cellular Data Operation}


\section{Challenges}

\subsection{SIM7020E Module}

\subsection{Memory Constraints and SSL Certificates}

\subsection{NFC Reader Unresponsiveness}

\subsection{Storage}
