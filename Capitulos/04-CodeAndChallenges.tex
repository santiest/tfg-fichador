\chapter{Code Implementation and Development Challenges}
\label{cap:codeAndChallenges}

The software development phase represents a significant portion of the project timeline, demanding 
meticulous attention to detail and thorough problem-solving skills. In the forthcoming sections, 
an overview of the software requirements will be provided, shedding light on how these 
requirements were effectively addressed. Additionally, detailed insights into the encountered 
challenges and the corresponding solutions devised will be explored.

One crucial decision made early on was the selection of CircuitPython as the programming language 
for the device. This decision stemmed from several key factors, including its open-source nature, 
extensive library support, and the availability of pre-existing libraries with MIT licenses, 
particularly those essential for handling NFC readers. Furthermore, CircuitPython's user-friendly 
syntax and simplified microcontroller programming paradigm were deemed advantageous compared to 
alternatives like MicroPython, contributing to smoother development workflows.

However, it's important to note that despite the benefits of CircuitPython, the project encountered 
limitations imposed by the microcontroller's memory capacity, capped at 256kB. This constraint was 
exacerbated by the use of Python, a high-level language known for its memory-intensive nature. 
Throughout the development process, various memory optimization strategies were implemented to 
mitigate these limitations. These optimizations will be thoroughly explained, providing valuable 
insights into managing resource constraints in microcontroller-based projects.

In addition to the aforementioned aspects, it's crucial to highlight the iterative nature of 
software development, where frequent testing, debugging, and refinement cycles were integral to 
achieving desired functionality and performance. This iterative approach enabled the project team 
to address emerging issues promptly and iteratively enhance the software's robustness and 
reliability.

Furthermore, a detailed examination of the software architecture and design decisions made will 
offer valuable insights into the project's development methodology and the rationale behind 
specific implementation choices.

\section{CircuitPython: Advantages and Disadvantages}

CircuitPython is an open-source programming language designed specifically for 
microcontroller-based development. Developed primarily by Adafruit Industries, CircuitPython is 
built on top of the Python programming language, offering a simplified yet powerful platform for 
programming microcontrollers. Unlike traditional embedded programming languages, CircuitPython 
abstracts many low-level complexities, making it more accessible to beginners and hobbyists while 
still providing advanced features for experienced developers.

\begin{figure}[h]
	\centering
	\includegraphics[width = .5\textwidth]{Imagenes/Vectorial/circuitpython_logo.pdf}
	\caption{CircuitPython's logo}
	\label{fig:circuitpython_logo}
\end{figure}

One of the key advantages of CircuitPython is its user-friendly syntax and high-level abstractions, 
which resemble those of Python, a language renowned for its readability and simplicity. This makes 
CircuitPython particularly attractive to beginners and educators, enabling them to quickly grasp 
programming concepts and develop projects without being bogged down by intricate syntax or complex 
setup procedures.

Furthermore, CircuitPython simplifies the process of interacting with hardware peripherals and 
sensors commonly used in embedded systems. It provides a consistent and intuitive API (Application 
Programming Interface) for accessing GPIO pins, I2C and SPI interfaces, analog inputs, and other 
hardware features, streamlining the development process and reducing the learning curve associated 
with embedded programming.

Another notable advantage of CircuitPython is its extensive library support, particularly for 
popular microcontroller boards and peripherals. Adafruit maintains a vast repository of 
CircuitPython libraries, covering a wide range of sensors, displays, actuators, communication 
modules, and other components commonly used in electronics projects\footnote{Adafruit's Library 
Bundle: \url{https://github.com/adafruit/Adafruit_CircuitPython_Bundle}}. These libraries abstract 
the complexities of interfacing with specific hardware, allowing developers to focus on application 
logic rather than low-level hardware details.

Despite its numerous advantages, CircuitPython does have some limitations and drawbacks. One 
notable limitation is its higher memory footprint compared to lower-level languages like C or 
assembly, which can be a concern for projects with strict memory constraints. Additionally, while 
CircuitPython abstracts many hardware complexities, it may sacrifice some performance compared to 
bare-metal or lower-level programming approaches.

Overall, CircuitPython offers a compelling combination of simplicity, accessibility, and 
versatility for microcontroller-based development, making it an excellent choice for a wide range 
of projects, from educational endeavors to commercial products. Its rich ecosystem of libraries, 
ease of use, and strong community support contribute to its popularity among hobbyists, educators, 
and professional developers alike\cite{circuitpython_docs}.

For this project, CircuitPython version 9.0.0 will be used.

\section{Code Requirements}

The device must prioritize responsiveness and intuitive usability for all employees. This ensures 
that the device can be easily operated by users of varying technical proficiency levels, promoting 
efficient and accurate clocking processes.

Furthermore, the device must maintain the highest possible uptime to ensure continuous operation 
without interruptions. This is crucial for accurate time tracking and prevents any disruptions that 
could lead to discrepancies in employee attendance records.

Data integrity is of paramount importance, as any loss of data could result in unfair deductions 
from employees' wages. Therefore, the device must be designed to prevent data loss under any 
circumstances, even in the event of power outages or connectivity issues. It should have the 
capability to store clockings locally and reliably transmit them to the server once connectivity is 
restored.

In situations where internet connectivity is temporarily lost, the device must continue to store 
clockings and transmit them as soon as connectivity is regained. This ensures that no clocking data 
is lost and enables seamless operation regardless of network conditions.

Efficient data transmission is essential for real-time monitoring and analysis of employee 
attendance. Therefore, the device should send clocking data as soon as it is available, without any 
delays, to ensure timely reporting and accurate statistics.

Flexibility is also key, as the device must be capable of operating using either WiFi or cellular 
data connectivity, depending on the specific configuration and deployment requirements.

Moreover, the device should incorporate a mode-switching feature, allowing it to display relevant 
information such as the device's ID and enabling easy implementation of additional modes in the 
future. This ensures scalability and adaptability to changing needs and requirements.

Additionally, the device must display the current time on the screen to provide users with 
real-time information and facilitate accurate clocking processes.

To enhance user experience and ensure successful clocking operations, the device should provide 
visual and audible cues to indicate successful clockings. It should also alert users to errors, 
such as attempting to clock in multiple times with the same keycard.

Finally, all clocking data should be securely transmitted to an API using POST requests, ensuring 
that it is reliably delivered to the server for processing and storage. This maintains data 
integrity and enables seamless integration with existing systems and workflows.


\section{Code Implementation}

\subsection{Asynchronous Programming}

Asynchronous programming is a programming paradigm that allows multiple tasks to be executed 
concurrently, without the need for explicit parallelism or multithreading. It enables programs to 
perform non-blocking operations, where tasks can be started and completed independently of each 
other, leading to improved performance and responsiveness.

In CircuitPython, asynchronous programming is achieved using the \texttt{asyncio} module, which 
provides a framework for writing asynchronous code using coroutines. Coroutines are functions 
that can be paused and resumed during execution, allowing for efficient task scheduling and 
coordination.

The key concept in asynchronous programming in Python is the \texttt{async} and \texttt{await} 
keywords. Functions defined with the \texttt{async} keyword are called asynchronous functions, and 
they can be paused with the \texttt{await} keyword when they encounter an operation that may block, 
such as I/O operations or network requests.

When an asynchronous function is awaited, it returns control to the event loop, allowing other 
tasks to continue executing in the meantime. Once the awaited operation completes, the function 
resumes execution from the point where it was paused.

The event loop is a central component of asynchronous programming in Python and CircuitPython. It 
manages the execution of asynchronous tasks and ensures that they are scheduled and executed 
efficiently. The event loop continuously checks for tasks that are ready to run, executes them 
until completion or until they are paused, and then moves on to the next task 
\cite{circuitpython_docs}.

The code extensively utilizes asynchronous programming to enhance the perception of concurrency 
and prevent the device from becoming unresponsive. This is crucial, especially considering that the 
device displays the time, including seconds, on the screen. However, while this approach offers 
significant benefits, it also introduces certain drawbacks, which will be addressed later.

Later on, this aspect will be further explored. Essentially, two tasks will run within the event 
loop: one managing the main loop, waiting for workers to approach the device with their keycards, 
while the other will handle the transmission of clockings to the internet.

\subsection{Main Loop and Operating Modes}

The main loop executes functions on the current operating mode. These modes, essentially Python 
classes, implement an interface containing four functions:\\ \texttt{enter\_mode}, 
\texttt{exit\_mode}, \texttt{card\_received}, and \texttt{no\_card\_found}. Depending on the event, 
one or more of these functions may be invoked.

Initially, the device starts in a default mode. When the ``configuration keycard'' is detected by 
the NFC reader, the mode transitions to a different one, triggering the \texttt{exit\_mode} 
function on the mode that is being exited, and \texttt{enter\_mode} in the mode that entered 
execution.

As part of its operation, the NFC reader is configured with a timeout to detect the presence of any 
card near its antenna. Upon successful detection, it reads the UID of the card.

Additionally, the main loop is responsible for sending a periodic ``keepalive'' message to the 
server to maintain connectivity.

\subsection{Interchangeable WiFi and Cellular Data Operation}

In certain deployment locations, access to WiFi may not always be available. Furthermore, even in 
areas with existing WiFi coverage, circumstances can change, requiring the device to be relocated 
to areas without such connectivity. In such scenarios, cellular data operation becomes essential.

To facilitate internet usage, an interface was created with fundamental functions for internet 
communication. Essentially, the primary requirement was the capability to send POST requests, or a 
series of them. Consequently, two classes were designed to adhere to this interface: one serving as 
a controller for WiFi connectivity, and the other for the SIM7020E module.

Either of these controllers could be employed seamlessly, with the device operating without any 
noticeable difference.

Regarding the WiFi functionality, Adafruit offers comprehensive code for executing requests via 
WiFi under an MIT license. This code was subsequently modified to operate asynchronously, aiding in 
preventing device hang-ups. Details regarding the SIM7020E code will be provided in the subsequent 
subsection.


\subsection{SIM7020E Module}

The SIM7020E module, which facilitates cellular data connectivity, operates using 
\textit{AT commands}.

AT commands, short for ``Attention commands'', are a standardized set of instructions used to 
communicate with modems, including those integrated into cellular modules like the SIM7020E. These 
commands are typically sent from a host device, such as a microcontroller or computer, to the modem 
via a serial communication interface (such as \textit{UART}, which is the case for this Waveshare 
board).

AT commands follow a specific syntax, starting with the characters ``AT'' (hence the name), 
followed by a command mnemonic and optional parameters. They are used to configure various 
settings, initiate actions (such as making a phone call, sending a text message or sending a POST 
request), query status information, and handle other modem-related tasks.

For example, to check the signal strength of a cellular connection, the command ``AT+CSQ'' may be 
sent to the modem, which would respond with a value indicating the signal quality. Similarly, to 
establish a data connection, the command ``AT*MCGDEFCONT'' may be used, followed by parameters 
specifying the APN (Access Point Name) and other connection details. It is worth noting that some 
of these commands may vary depending on the module used, and some are manufacturer specific 
\cite{sim7020e_atcommandsmanual}.

AT commands provide a standardized interface for interacting with modems, allowing developers to 
integrate modem functionality into their applications without needing to understand the intricacies 
of the underlying hardware.

SIMCOM offers a specific manual for the SIM7020E, delineating the AT commands, their parameters, 
usage instructions, and responses. This manual proved indispensable for using the module, 
especially considering the scarcity of code examples available online for this specific module, 
apart from the brief code snippets provided by Waveshare as illustrative examples.

Commands are transmitted to the transmit UART buffer, where they await processing by the SIM7020E 
module. Following a command transmission, the module responds, and its responses are stored in the 
designated memory region allocated for the receive UART buffer.

Subsequently, the receive UART buffer must be periodically checked to ascertain whether the entire 
response has been received. This periodic monitoring is essential since not all responses may be 
promptly received without any delay, potentially leading to device hang-ups. This necessity for 
asynchronous programming arose primarily from this requirement. Consequently, with asynchronous 
programming implemented, a command is dispatched, and the response is continuously read until the 
entire message is received, allowing for subsequent commands to be processed.

It's crucial to note that only one ``thread'' is engaged in communication with the SIM7020E module, 
obviating the need for locks or synchronization mechanisms to prevent the dispatch of additional 
commands while awaiting a response. Such simultaneous command dispatch would risk causing errors 
within the module.

Certain commands, particularly those related to establishing connections with the internet and 
external servers, inherently entail longer execution times, necessitating longer wait times during 
the response retrieval process.

Regarding POST requests, a sequential series of steps must be meticulously followed to ensure 
successful transmission. These steps will be explained to provide a clearer understanding of the 
module's interaction mechanisms.

\subsubsection*{CHTTPCREATE}
First, an HTTP or HTTPS client instance must be created within the module. Here, the host is 
specified:
\begin{center}
	\texttt{AT+CHTTPCREATE="https://example.com"}
\end{center}
And after waiting for the corresponding response, the following will be read:
\begin{center}
	\texttt{+CHTTPCREATE: 0\textbackslash r\textbackslash n\textbackslash r\textbackslash
	nOK\textbackslash r\textbackslash n}
\end{center}

Notice the appearance of ``\texttt{\textbackslash r}'' and ``\texttt{\textbackslash n}'', due to 
the use of Python's byte strings.

Most commands end with the appearance of ``\texttt{OK\textbackslash r\textbackslash n}'', which 
allows for easily checking if a command has finished executing, to then send the following one.

\subsubsection*{CHTTPCON}
The next step is to establish the HTTP connection with the server, by executing the following:
\begin{center}
	\texttt{AT+CHTTPCON=0}
\end{center}

Notice the ``0'' used in the command. That number has to be the same one as the one returned as a 
response in the \texttt{CHTTPCREATE} command. What that number means, is the slot that the host is 
occupying in the module's memory. In the SIM7020E, there can be only 5 hosts in total (numbers 0 
to 4). Thus, by connecting to the index ``0'', the device will establish the connection with the 
host just created.

The following response will be received in case of success:
\begin{center}
	\texttt{\textbackslash r\textbackslash nOK\textbackslash r\textbackslash n}
\end{center}

\subsubsection*{CHTTPSEND}
Now, the actual POST request can be sent. The structure of the command is the following: 
``\texttt{AT+CHTTPSEND=<httpclient\_id>, <method>, <path>, <header>,\\<content\_type>, 
<content\_string>}''.

The values for this command will be the following:
\begin{itemize}
	\item \texttt{httpclient\_id}: ``0'', as explained before.
	\item \texttt{method}: the number ``1'' corresponds to the POST method, as explained by the 
	commands manual.
	\item \texttt{path}: this is the endpoint. The host's URL must not be specified.
	\item \texttt{header}: the header contents, encoded in hexadecimal.
	\item \texttt{content\_type}: the content type, as in any regular POST request. For example: 
	``application/json''.
	\item \texttt{content\_string}: the data contained in the POST request, encoded in hexadecimal.
\end{itemize}

In the following example, the following headers will be sent, encoded in hexadecimal:

\begin{tabular}{l}
    \texttt{Accept: */*} \\
    \texttt{Connection: Keep-Alive} \\
    \texttt{User-Agent: RPI\_PICO}
\end{tabular}

And the following data:

\begin{center}
	\texttt{\{"data": "some information"\}}
\end{center}

This is the resulting command:

\begin{tabular}{l}
    \texttt{AT+CHTTPSEND=0,1,"/some\_endpoint",} \\
	\texttt{4163636570743a202a2f2aa436f6e6e656374696f6e3a204b65657} \\
	\texttt{02d416c697665a557365722d4167656e743a205250495f5049434fa,} \\
	\texttt{"application/json",} \\
	\texttt{7b2264617461223a22736f6d6520696e666f726d6174696f6e227d}
\end{tabular}

\subsubsection*{CHTTPDISCON}
The last command to be executed is the disconnection from the HTTP server:
\begin{center}
	\texttt{AT+CHTTPDISCON=0}
\end{center}

The number sent in the command is a ``0'' again, and the response will be another OK. With this, 
the POST request will have been sent, and the connection terminated.

The HTTP field could be destroyed with the \texttt{CHTTPDESTROY} command, but this is not needed, 
since it will be reused, and would only increase inefficiencies in sending a new request. In fact, 
if there are many POST requests queued to be sent, then they could even be sent continuously before 
sending the \texttt{CHTTPDISCON}.

\section{Challenges}

This section examines the challenges faced during the development process. From technical obstacles 
to unforeseen setbacks, each hurdle provided an opportunity for problem-solving and refinement.

\subsection{Lack of Multithreading}

While CircuitPython incorporates asynchronous programming via the \texttt{asyncio} library, it's 
essential to note that this differs from traditional multithreading, a feature that is in fact 
supported by MicroPython. Asynchronous programming with \texttt{asyncio} allows for concurrent 
execution of tasks, but it operates within a single thread, managing task switching based on 
input/output operations or explicit yields. In contrast, multithreading involves the simultaneous 
execution of multiple threads, each running independently and potentially executing different code 
segments concurrently.

One significant challenge arises from the nature of CircuitPython's asynchronous programming model. 
While \texttt{asyncio} allows for the concurrent execution of tasks within a single thread, certain 
code segments lacking explicit yields may monopolize the event loop, hindering task switching and 
potentially causing sluggish performance. This becomes particularly evident in scenarios where 
tasks, such as establishing a WiFi connection, encounter delays. In such cases, where CircuitPython 
lacks native support for asynchronous WiFi operations, the device may become unresponsive for 
extended periods, up to 15 seconds, before timing out.

Regrettably, there exists no straightforward solution to address this challenge within the current 
framework. A potential avenue for improvement involves migrating the codebase to MicroPython, which 
offers multithreading capabilities. However, this approach necessitates significant reworking of 
existing libraries, presenting a formidable task exacerbated by time and budget constraints.

\subsection{Memory Constraints and SSL Certificates}

Python's inherent memory footprint, compared to C, is often larger, which poses a challenge given 
the limited 256kB of RAM available. Upon loading CircuitPython during startup, approximately 120kB 
of memory remains free. However, once all libraries are loaded and devices initialized, this 
dwindles to around 60kB.

While this may seem sufficient, complications arise when attempting to send a POST request to an 
HTTPS server over WiFi. SSL certificate validation requires a minimum of 60kB of available memory, 
dangerously close to the device's operational limit. The lack of descriptive error messages 
compounds this issue, necessitating extensive investigation to decipher the underlying problem. 
Anecdotal evidence suggests that errors may occur when available memory dips below 42kB 
\cite{ssl_memoryerror}, though during testing, errors consistently manifested when nearing the 60kB 
threshold. This discrepancy may be attributed to various factors, including CircuitPython version 
and memory fragmentation.

To mitigate memory constraints, a strategy was devised to import libraries only when needed. For 
instance, if the device operates in WiFi mode, code related to the SIM7020E module remains dormant. 
Additionally, the garbage collector is called upon constantly to minimize memory fragmentation and 
ensure sufficient space for SSL certificates.

During testing, the following errors were encountered:

\begin{itemize}
	\item When available memory is insufficient by a big margin:
	
	\begin{tabular}{l}
		\texttt{x509-crt-bundle:PK verify failed with error FFFFBD70} \\
		\texttt{x509-crt-bundle:Failed to verify certificate} \\
		\texttt{Exception: <class 'RuntimeError'> Error connecting socket:} \\
		\texttt{ (-12288, 'MBEDTLS\_ERR\_X509\_FATAL\_ERROR')}
	\end{tabular}

	\item When available memory is nearly adequate but still falls short a regular 
	\texttt{MemoryError}.
\end{itemize}

In testing, it was challenging to ascertain the precise threshold between encountering the two 
errors. However, observations indicated that the first error surfaced at approximately 50kB or 
below, while a MemoryError occurred within the range of 50kB to 60kB.

Addressing this challenge proves daunting, as memory optimization can only achieve so much, and the 
alternative—removing code to reduce functionality—is undesirable.



\subsection{NFC Reader Unresponsiveness}

\subsection{Storage}

\subsection{Connecting to the Google Cloud Platform}


\section{Overview}
