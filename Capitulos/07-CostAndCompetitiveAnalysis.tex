\chapter{Cost Analysis and Competitive Comparison}
\label{cap:costAndCompetitiveAnalysis}

The primary motivation for undertaking this project was to significantly reduce costs associated 
with the previous solution, which relied on mobile phones. This chapter will provide a detailed 
analysis of the costs incurred during the development and implementation of the new device, 
alongside a comparison with the costs of the previous mobile phone-based solution. By examining 
both the financial aspects and the competitive landscape, this analysis aims to demonstrate the 
economic advantages of the newly developed system and its potential for broader application.

\section{The Previous Solution}

The previous solution for processing employee clock-ins involved using mobile phones mounted on 
walls, equipped with an application from a software provider designed for this purpose. While the 
application itself functioned effectively, the hardware presented significant issues. These 
phones, which were constantly plugged into power sources and often exposed to direct sunlight, 
suffered from severe battery degradation. Within six months, the batteries would swell and become 
unusable, necessitating frequent replacements.

Even if these phones were on the lower-end of performance, their cost was not negligible, and with 
the need for replacements every six months, the expenses quickly escalated. Additionally, the 
mounting cases used for the phones were notably brittle and unable to withstand impacts from 
falls. Despite their cost not being nearly as high as the phones, they were too costly for the 
purpose they served, adding to the overall expense and contributing to the recurring costs of 
maintaining the system.

This combination of frequent hardware failures and the associated costs highlighted the need for a 
more durable and cost-effective solution, prompting the development of the new device detailed in 
this project.

\section{The Proposed Solution}

The proposed solution aims not only to significantly reduce costs but also to greatly enhance the 
longevity of the system. The new device is designed with durability and modularity in mind, 
ensuring that it can withstand prolonged use and adverse conditions far better than the previous 
phone-based system.

It is estimated that the electronics in the device should last at least three years, a substantial 
improvement over the six-month lifespan of the phones previously used. This extended longevity 
translates into fewer replacements and lower long-term costs. Additionally, in the event of a 
component failure, the modular design allows for the replacement of the specific faulty part 
rather than discarding the entire device. This repairability further contributes to cost savings 
and reduces electronic waste.

By addressing both the economic and operational shortcomings of the previous system, the proposed 
solution offers a more sustainable and efficient approach to managing employee clock-ins.

\subsection{Pricing}

To maintain confidentiality regarding proprietary costs, this analysis will focus exclusively on 
the prices of publicly available parts. For example, while custom PCB costs are excluded due to 
their sensitive nature, prices for readily available components such as the Raspberry Pi Pico, NFC 
reader, and 3D printing materials will be included. This approach ensures that the analysis 
remains transparent and informative without compromising any sensitive financial information.

The table with approximate pricing for each component at the time of publication can be seen on 
Table \ref{tab:pricing_table}. Consider that the enclosure's weight is significantly less than 1 
kilogram, and that costs related to assembling the device and developing software are not being 
included.


\begin{table}
	\centering
	\begin{tabular}{c|c}
		\textbf{Component} & \textbf{Approximate Cost} \\
		\hline\hline
		Raspberry Pi Pico WH & 8€\\
		Waveshare SIM7020E Board & 20€\\
		NFC Reader & 7€\\
		LCD Display & 8€\\
		1kg of PETG & 20€\\
		\hline
	\end{tabular}
	\caption{Pricing table}
	\label{tab:pricing_table}
\end{table}
